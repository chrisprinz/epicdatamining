\documentclass[]{scrartcl}

\usepackage{mathtools}
\renewcommand{\familydefault}{\sfdefault}
\usepackage{color}
\usepackage{listings}
\definecolor{pblue}{rgb}{0.13,0.13,1}
\definecolor{pgreen}{rgb}{0,0.5,0}
\definecolor{pred}{rgb}{0.9,0,0}
\definecolor{pgrey}{rgb}{0.46,0.45,0.48}
\setlength{\parindent}{0pt}

\lstset{language=Java,
	showspaces=false,
	showtabs=false,
	breaklines=true,
	showstringspaces=false,
	breakatwhitespace=true,
	commentstyle=\color{pgreen},
	keywordstyle=\color{pblue},
	stringstyle=\color{pred},
	basicstyle=\footnotesize
}

%opening
\title{Intelligent Data Management - Exercise 3}
\author{Christoph Prinz}

\begin{document}

\maketitle

\subsection*{Assignment 1}

Jaccard Similarities formula = $|S \cap T| \div |S \cup T|$ \\

a) $SIM(\{1, 2, 3, 4\}, \{2, 3, 5, 7\}) = \frac{1}{3}$\\\\
$SIM(\{1, 2, 3, 4\},\{2, 4, 6\}) = \frac{2}{5}$\\\\
$SIM(\{2, 3, 5, 7\}, \{2, 4, 6\}) = \frac{1}{6}$\\\\

b) $SIM(\{1, 1, 1, 2\}, \{1, 1, 2, 2, 3\}) = \frac{2}{3}$\\\\
$SIM(\{1, 1, 1, 2\}, \{1, 2, 3, 4\}) = \frac{1}{2}$\\\\
$SIM(\{1, 1, 2, 2, 3\}, \{1, 2, 3, 4\}) = \frac{3}{4}$\\

\subsection*{Assignment 2}

a) $\{$The, most, effective$\}$, $\{$most, effective, way$\}$, $\{$effective, way, to$\}$, $\{$way, to, represent$\}$, $\{$to, represent, documents$\}$, $\{$represent, documents, as$\}$, $\{$documents, as, sets$\}$, $\{$as, sets, for$\}$, $\{$sets, for, the$\}$, $\{$for, the, purpose$\}$ \\


b) Stopwords are marked as \textit{italic}: "\textit{The} most effective \textit{way} \textit{to} represent documents \textit{as} sets,\textit{ for the} purpose \textit{of} identifying lexically similar documents \textit{is to} construct from \textit{the} document \textit{the set of} short strings that appear within \textit{it}."\\

Resulting Shingles:\\

$\{$The, most, effective$\}$,$\{$way, to, represent$\}$, $\{$to, represent, documents$\}$, $\{$as, sets, for$\}$, $\{$for, the, purpose$\}$, $\{$the, purpose, of$\}$, $\{$of, identifying, lexically$\}$, $\{$is, to, construct$\}$, $\{$to, construct, from$\}$, $\{$the, document, the$\}$, $\{$the, set, of$\}$, $\{$set, of, short $\}$, $\{$of, short, strings$\}$


 

%
%
%
%
%%\subsection*{Exercise a}
%%
%%Term $i$ appears in $n_i$ of the $N$ documents \\
%%
%%Appears in 40 documents:
%%$IDF_i = log_2(\frac{N}{n_i}) = log_2(\frac{10000000}{40}) = 18$\\
%%
%%Appears in 10000 documents:
%%$IDF_i = log_2(\frac{N}{n_i}) = log_2(\frac{10000000}{10000}) = 10$ \\
%%
%%
%%\subsection*{Exercise b}
%%
%%Given the occurrence of a term $i$ in document $j$ is $f_{ji}$ and $max_k f_{kj}$ the maximum number of occurrences of any term in this document, the term frequency $TF$ is defined as:\\
%%
%%$TF_{ij} = \frac{f_{ij}}{max_k f_{kj}}$\\
%%
%%Word $w$ appears in 320 documents. In document $d$ the maximum occurence of any word is 15. \\
%%
%%a) $w$ appears once:\\
%%\begin{addmargin}[25pt]{0pt} 
%%
%% $TF_{wd} = \frac{1}{15}$\\\\
%% $IDF_w = * log_2\frac{10000000}{320} = 15$\\\\
%% $TF.IDF = \frac{1}{15} * 15 = 1$\\\\
%%
%%\end{addmargin}
%%
%%b) $w$ appears five times: \\
%%\begin{addmargin}[25pt]{0pt} 
%%$TF_{wd} = \frac{5}{15} = \frac{1}{3}$\\\\
%%$IDF_w = * log_2\frac{10000000}{320} = 15$\\\\
%%$TF.IDF = \frac{1}{3} * 15 = 5$\\\\
%%\end{addmargin}
%%
%%\subsection*{Exercise c}
%%
%%The basic rule is, that the possible hash-keys should not have any common factor with B (15 in this example). The population should therefore not be generated with a $c$ of 3 or 5. To achieve an equal distribution of the generated hashs, $c$ should be set to 1.
%%
%%\section*{Assignment 2}
%%
%%\subsection*{Exercise a}
%%
%%The implementation consists out of a class \texttt{Document}, that counts words in a document and calculates the term-frequency of a given term in this document. The class \texttt{DocumentCollection} reads multiple input files and calculates IDF and TF.IDF for a given document and a given term. The source code is printed on the following pages.
%
%%\subsection*{Exercise b}
%%The choice of parameters is explained with comments in the source code.
%
%
%\section*{Assignment 1: Bonferroni's Principle}
%
%
%a) days with observation = 2000
%	
%\[\text{number of pairs of days} = \binom{2000}{2} \approx 2 \times 10^6 \]
%\[\text{number of suspected pairs} = 5 \times 10^{17} \times 2 \times 10^6 \times 10^{-18} = 1,000,000 \]\\
%
%	
%	
%b) The number of people observed was raised to 2 billion (and there were therefore 200,000 hotels).
%	
%\[	\text{number of pairs of people} = \binom{2 \times 10^9}{2} \approx 2 \times 10^{18} \]
%
%\[ \text{chance same hotel} = \frac{10^-4}{2 \times 10^5} = 5 \times 10^{-10} \]
%
%\[\text{chance same hotel on two different given days} = (5 \times 10^{-10})^2 = 2.5 \times 10^{-19} \]
%
%\[ 	\text{ suspected pairs} = 2 \times 10^{18} \times 5 \times 10^5 \times 2.5 \times 10^{-19} = 250,000 \]\\
%
%c) We only reported a pair as suspect if they were at the same hotel at the same time on three different days.
%		
%\[\text{chance same hotel on three different days} = (10^{-9})^3 = 10^{-27} \]
%
%\[\text{number of "triples" of days} =  \binom{1000}{3} \approx 1.7 \times 10^8 \]
%
%\[\text{suspected pairs} = 5 \times 10^{17} \times 1.7 \times 10^8 \times 10^{-27} = 0.085 \]
%
%\section*{Assignment 2: Base of the natural logarithm}
%
%a) Approximations in terms of $e$
%
%\[(1.01)^{500} = (1 + 0.01)^{500} = e^{0.01 \times 500} = e^5 \]
%\[(1.05)^{1000}    = (1 + 0.05)^{1000} = e^{0.05 \times 1000} = e^{50} \]
%\[(0.9)^{40} = (1 - 0.1)^{40} = e^{-0.1 \times 40} = e^{-4} \]
%
%	
%b) Approximation of $e^x$ with Taylor expansion:
%		
%\[e^{1/10} \approx 1 + \frac{1}{10} + \frac{1}{200} + \frac{1}{6000} + \frac{1}{240000} \approx 1.105 \]
%	
%\[e^{-1/10} \approx 1 - \frac{1}{10} + \frac{1}{200} - \frac{1}{6000} + \frac{1}{240000} \approx 0.904 \]
%	
%\[e^2 \approx 1 + 2 + \frac{4}{2} + \frac{8}{6} + \frac{16}{24} \approx 7.000 \]
	


\end{document}
