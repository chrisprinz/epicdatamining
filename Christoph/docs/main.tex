\documentclass[]{scrartcl}

\usepackage{mathtools}
\renewcommand{\familydefault}{\sfdefault}
\usepackage{color}
\usepackage{listings}
\definecolor{pblue}{rgb}{0.13,0.13,1}
\definecolor{pgreen}{rgb}{0,0.5,0}
\definecolor{pred}{rgb}{0.9,0,0}
\definecolor{pgrey}{rgb}{0.46,0.45,0.48}

\lstset{language=Java,
	showspaces=false,
	showtabs=false,
	breaklines=true,
	showstringspaces=false,
	breakatwhitespace=true,
	commentstyle=\color{pgreen},
	keywordstyle=\color{pblue},
	stringstyle=\color{pred},
	basicstyle=\footnotesize
}

%opening
\title{Intelligent Data Management - Exercise 1}
\author{Christoph Prinz}

\begin{document}

\maketitle


\section*{Assignment 1}

\subsection*{Exercise a}

Term $i$ appears in $n_i$ of the $N$ documents \\

Appears in 40 documents:
$IDF_i = log_2(\frac{N}{n_i}) = log_2(\frac{10000000}{40}) = 18$\\

Appears in 10000 documents:
$IDF_i = log_2(\frac{N}{n_i}) = log_2(\frac{10000000}{10000}) = 10$ \\


\subsection*{Exercise b}

Given the occurrence of a term $i$ in document $j$ is $f_{ji}$ and $max_k f_{kj}$ the maximum number of occurrences of any term in this document, the term frequency $TF$ is defined as:\\

$TF_{ij} = \frac{f_{ij}}{max_k f_{kj}}$\\

Word $w$ appears in 320 documents. In document $d$ the maximum occurence of any word is 15. \\

a) $w$ appears once:\\
\begin{addmargin}[25pt]{0pt} 

 $TF_{wd} = \frac{1}{15}$\\\\
 $IDF_w = * log_2\frac{10000000}{320} = 15$\\\\
 $TF.IDF = \frac{1}{15} * 15 = 1$\\\\

\end{addmargin}

b) $w$ appears five times: \\
\begin{addmargin}[25pt]{0pt} 
$TF_{wd} = \frac{5}{15} = \frac{1}{3}$\\\\
$IDF_w = * log_2\frac{10000000}{320} = 15$\\\\
$TF.IDF = \frac{1}{3} * 15 = 5$\\\\
\end{addmargin}

\subsection*{Exercise c}

The basic rule is, that the possible hash-keys should not have any common factor with B (15 in this example). The population should therefore not be generated with a $c$ of 3 or 5. To achieve an equal distribution of the generated hashs, $c$ should be set to 1.

\section*{Assignment 2}

\subsection*{Exercise a}

The implementation consists out of a class \texttt{Document}, that counts words in a document and calculates the term-frequency of a given term in this document. The class \texttt{DocumentCollection} reads multiple input files and calculates IDF and TF.IDF for a given document and a given term. The source code is printed on the following pages.

\subsection*{Exercise b}
The choice of parameters is explained with comments in the source code.

\end{document}
