\documentclass{article}
\usepackage[utf8]{inputenc}
\usepackage[english, croatian]{babel}
\usepackage{listings}

\title{Intelligent Data Mining - Exercise 1}
\author{Michael Debono Mrđen}
\date{18 October 2017}

\begin{document}

\maketitle

\section{Assignment 1: Introducing TF.IDF and hash functions}
\renewcommand{\labelenumi}{\alph{enumi}.}
\renewcommand{\labelenumii}{(\alph{enumii})}
\begin{enumerate}
\item
\begin{enumerate}
\item{40 documents

$IDF \\ = log_2(10,000,000/40) \\ = 18$}

\item{10,000 documents

$IDF \\ = log_2(10,000,000/10,000) \\ = 10$}
\end{enumerate}
\item
\begin{enumerate}
\item{once

$TF_{wd} = 1/15$

$IDF_w = log_2(10,000,000/320) \\ = 15$

$TF.IDF = 1/15 * 15 \\ = 1$}

\item{five times

$TF_{wd} = 5/15 \\ = 1/3$

$TF.IDF = 1/3 * 15 \\ = 5$}
\end{enumerate}
\item{The hash function $h$ will be suitable if $c$ is not equal to the factors of 15, i.e. 3 and 5.}
\end{enumerate}
\section{Assignment 2: Implementation}
\renewcommand{\labelenumi}{\alph{enumi}.}
\renewcommand{\labelenumii}{(\alph{enumii})}
\begin{enumerate}
\item{Main.java: \\
\lstset{breaklines=true}
\lstinputlisting[language=Java]{Main.java}
}
\item{For the implementation, documents are considered to be a String and a set of documents an array of String.

The method Main.tf() takes a word and a document, both as String's, as the TF function is the frequency of a term or word divided by the maximum frequency of all terms in one document.

The method Main.idf() takes a word as String and a set of documents as an array of String. The IDF function is based on the number of documents in which a term or word appears, which explains the choice of inputs.}
\end{enumerate}

\end{document}
